\section{Motivation}
\label{sec:motivation}

It is very difficult for an administrator to gain a full understanding of exactly what flow rules get inserted into a switch. 
Applications can be custom made by groups within an enterprise or from some other third party.
With many applications needing to be run on a network, it isn't realistic for an administrator to know every detail of every application that is being run.
There is no guarantee that any two applications, especially when created by a third party, don't logically overwrite each other's rules by accident.
OpenFlow 1.0 specifies that switches must have support to detect and send an error message when there is an attempt to install a flow rule that has the same or overlapping headers as a rule currently in the flow table.
This is only done when the \texttt{ADD} request to install the rule has the \texttt{CHECK\_OVERLAP} flag set.
Otherwise, unless the rules have exactly the same headers, the switch will allow the new rule to be installed "side-by-side" with the old rule.
When packets come into the switch, they will take the action of the most specific rule that they can match in the flow table.
If the two rules have exactly the same headers than the new rule will overwrite the old rule.

% talk about how complicated it is for an admin to go from seeing something isn't working right in the network
Flow rules are installed in switches dynamically and are dependent on the traffic flowing through the network.
This can make it extremely difficult for an administrator to investigate undesirable behavior in their network.
The difficulty is exacerbated by the fact that rules timeout on a frequently.
By the time an administrator inspects the flow table of a switch, its state could already have already changed.

Thus, it is extremely useful for an admistrator to know when two or more applications are installing rules which logically conflict with each other.
Since all rules are written to the switches using the controller, it is the controller which has a world view of all the flow tables.
This allows the controller to make the decision if a rule should be allowed to be written to a switch or disallowed and flagged for administrator attention.
An administrator can debug unintended network behavior faster and resolve conflicts between separate applications.

% Move on the an illustration of a malicious app, talk about the firewall application and how to subvert it, using Floodlight
% in a large network, different groups may want to handle the traffic differently, this would be when they could get the ability to run their own app
% use a campus network for example, each department wants to control its own traffic and have its own application to do it

While unintentional misconfiguration of applications running on a controller are a definite possibility, there is also the case where an application author is being intentionally malicious.
We discuss one such malicious scenario next.

\subsection{Dynamic Flow Tunneling}
% Need more info about dft
As an illustration, we created a simple scenario of dynamic flow tunneling where a malicous application effectively bypasses a firewall rule to drop traffic between two hosts.
Figure {figure number} depicts our exmaple topology.
While this is a somewhat contrived situation (due to the limited number of hosts and simplicity of the applications), the ideas can be extended to larger, real world applications.

The first application acts like a simple "firewall" and installs a static rule in every switch to drop any traffic from \texttt{h1} to \texttt{h8} on port \texttt{80}.
The second application subverts the static rule by dynamically remapping \texttt{h1}'s flow to be redirected towards \texttt{h8}.
This is done by having \texttt{h1} send packets towards a dummy host that is connected to the same switch as \texttt{h8} using port \texttt{80} (the port number is arbitrary, it could be any high number port as long as the subversive application and the sender are in agreement).
When \texttt{h8}'s switch sees packets destined for the dummy host it dynamically installs a rule to remap the destination of those packets to \texttt{h8} on port \texttt{80}.
Since a flow can only match a single rule in a flow table, the switch remaps the packets' destination and forwards them on to \texttt{h8} without ever checking them against the firewall's static rule.
As with network debugging, the controller has a complete view of network and is therefore in a position to arbitrate which rules should be installed on a switch and which should be disallowed due to logical conflicts with existing rules.

