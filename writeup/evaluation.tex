\section{Evaluation}
\label{sec:evaluation}

We stress tested Floodlight-CD using the \texttt{pingall} command provided by Mininet.
This command goes through the host list and attempts to ping every other host in the topology.
\texttt{Pingall} was chosen since it requires that a new rule be written to the switch for both the request and reply of the majority of pings.
The only time a new rule is not written is if the rule already exists from a previous ping, i.e. when \texttt{h2} is pinging \texttt{h1} while the flow rules from \texttt{h1} pinging \texttt{h2} is still resident in the switch's flow table.
All flow rules in our tests were set with a 1 second hard timeout (the shortest Floodlight allows) to necessitate the writing of as many flows as possible to the switches.
This places a large load on Floodlight-CD to validate each rule as well as keep an up to date view of the switches' flow tables.
Since the installation of each dynamic rule is triggered by a packet being forwarded to the controller, the packets themselves must also be checked before getting written out to the switch.
Therefore, for every ping, the forward rule, the reverse rule, the request packet, and the response packet must be checked for the ping to succeed.

The tests were done with varying numbers of switches and hosts.
We timed how long it took to ping 
 
The results are shown in figure %TODO figure reference here

